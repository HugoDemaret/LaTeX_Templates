\section{2-coloriage}
\setcounter{subsection}{4}
\subsection{}
Montrons que $G$ est bipartite si et seulement si il est $2$-coloriable.\\
Nous appelerons nos couleurs $0$ et $1$, et les sous-ensemble (les deux parties) du graphe bipartite $A$ et $B$.
\begin{proof}
    (i) Si $G$ est $2$-coloriable, alors $G$ est bipartite.\\
    $G$ est $2$-coloriable donc : 
    $$
        \forall s \in S(G), couleur(s)=1 \vee couleur(s) = 0
    $$
    L'idée est ici de mettre tous les sommets coloriés en $0$ dans un ensemble $E$, et de mettre tous les sommets coloriés en $1$ dans un ensemble $F$. Il ne peut pas y avoir de sommet coloriés $0$ (resp. $1$) relié par une arête à un autre sommet $0$ (resp. $1$). Donc il n'existe pas de sommet de $E$ (resp. $F$) relié à un autre sommet de $E$ (resp. $F$). Les sommets sont donc dans deux ensembles disjoints. $G$ est donc bipartite.\\
    (ii) Si $G$ est bipartite, alors $G$ est $2$-coloriable.\\
    $G$ est bipartite donc :
    $$
        \forall s \in S(G), s \in S(E) \vee s \in S(F)
    $$
    De plus, d'après la définition de bipartie, toute arête a une extrémité dans $E$ et l'autre dans $F$, autrement dit ses sommets ne sont pas dans le même ensemble. Il suffit donc de colorier (arbitrairement) tous les sommets de $E$ en $0$ et tous ceux de $F$ en $1$. On a colorié tous les sommets du graphes en 2 couleurs, et c'est bien un coloriage car deux sommets d'un même ensemble (donc d'une même couleur) ne sont pas reliés.
    $G$ est donc $2$-coloriable.\\
    Cela conclu la preuve.
\end{proof}

\subsection{}
\lstinputlisting{parts/annexe/code/2-coloriage_6.cpp}