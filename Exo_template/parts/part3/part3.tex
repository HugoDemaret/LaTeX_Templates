\section{Algorithmes gloutons}
\setcounter{subsection}{6}
\subsection{}
On considère le graphe de Petersen (figure 1). On considère les ordres de numérotation suivants :
$$[1, 3, 4, 0, 2, 6, 5, 9, 8, 7]$$
$$[0, 7, 2, 5, 4, 6, 8, 1, 3, 9]$$
Donnons les coloriages correspondant, selon l'algorithme glouton décrit. 
Avec le premier ordre, on obtient le coloriage suivant : 
$(0, 0, 0, 0, 1, 1, 1, 2, 2, 2)$ et avec le second ordre on obtient le coloriage suivant : (0, 3, 0, 2, 1, 1, 1, 0, 2, 3).
Le premier coloriage est constitué de 3 couleurs, le second de 4 couleurs.


\subsection{}
\lstinputlisting{parts/annexe/code/algorithmesgloutons_8.cpp}


\subsection{}
\lstinputlisting{parts/annexe/code/algorithmesgloutons_9.cpp}

\subsection{}
Montrons que l'algorithme de coloriage glouton construit toujours un coloriage, et que celui-ci utilise au plus $d+1$ couleurs, avec $d$ le degré du graphe en entrée.\\
Nous montrerons d'abord que l'algorithme renvoie bien un coloriage, puis nous montrerons que le nombre de couleurs est au plus $d+1$.\\
(i)
Tout d'abord, observons que la fonction \texttt{min\_couleur\_possible} renvoie une couleur qui n'a préalablement pas été affectée aux voisins passé en paramètre (noté $k$ dans \texttt{glouton}). Donc, une couleur affectée à un sommet $k$ ne pourra être affectée à un voisin de $k$. Si $numerotation$ est bien une numérotation valide, alors tous les sommets auront été coloriés. $\texttt{glouton}$ renvoie donc bien un coloriage de $gphe$.\\
(ii)
On colore chaque sommet par une couleur non employée par ses voisins. Dans le pire cas, un sommet a $d$ voisins. On a donc $d$ couleurs utilisées (majoration), on ajoute donc une $d+1$ couleur et on colore notre sommet. Donc le coloriage possède au plus $d+1$ couleurs.

\subsection{}
Soit $G$ un graphe. Montrons que pour tout coloriage $L$ de $G$, il existe un ordre de numérotation des sommets tel que le coloriage glouton associé $L^\prime$ vérifie $L^\prime(s) \leq L(s)$ pour tout sommet $s$ de $G$.\\
Considérons une numérotation des sommets par couleur croissante. On sait que deux sommets de même couleur ne sont pas adjacents, donc l'appel de la fonction \texttt{min\_couleur\_possible} ne considèrera que les sommets adjacents à $s$ qui ont une couleur inférieure strictement à $s$. Nécessairement, $L^\prime(s)$, ne pourra pas être strictement supérieure à $L(s)$. On a donc $L^\prime(s) \leq L(s)$ pour tout $s$ dans $G$.\\
Montrons maintenant qu'il existe une numérotation de sommets tel que l'algorithme glouton renvoie un coloriage optimal.\\
Pour cela, on peut partir d'un coloriage optimal et numéroter les sommets en fonction de ce coloriage. Nécessairement cette numérotation est optimale.

\subsection{}
Dans cet algorithme, l'algorithme utilisé est un QuickSort. Celui-ci a une complexité moyenne $\comp{n \log n}$, et dans le pire cas $\comp{n^2}$. Un TimSort aurait été plus efficace. Un algorithme en $n^2$ est suffisant, l'algorihtme de Welsh-Powell étant quadratique dans tous les cas. On aurait donc pu se contenter d'un tri par insertion ou par selection. Il faut toutefois faire attention aux constantes multiplicative, car pour des données réelles celles-ci peuvent faire exploser le temps de calcul (par exemple, $0.5n^2$ vs $10^{100}n\log n$).
\lstinputlisting{parts/annexe/code/algorithmesgloutons_12.cpp}
