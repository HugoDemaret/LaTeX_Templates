\section{Algorithme de Wigderson}

\setcounter{subsection}{12}

Considérons un graphe $G$ avec $n$ sommets. Supposons que $G$ soit 3-coloriable, mais que l’on
ait cette information sans pour autant effectivement disposer d’un 3-coloriage de $G$. Trouver un
3-coloriage de $G$ pourrait prendre un temps exponentiel en n.
L’algorithme de Wigderson permet, pour un graphe $G$ supposé 3-coloriable, de trouver en
temps polynomial en $n$ un coloriage de $G$ avec $\comp{\sqrt{n}}$ couleurs (au sens où il existe $C > 0$ tel
que pour tout $n$ suffisamment grand, ce coloriage ait au plus $C\sqrt{n}$ couleurs). Cet algorithme repose sur la propriété suivante : 


\subsection{}
Soit $k>0$. Montrons que si $G$ est $(k+1)-$coloriable, alors pour tout sommet $s$ de $G$, le sous-graphe induit par $V(s)$ est $k-$coloriable.\\
Tout d'abord, rappelons la définition de sous-graphe induit.
\\
\begin{definition}[Sous-graphe induit]
    \'Etant donné un graphe $G = \brac{S,A}$, le sous-graphe induit $G^\prime$ par un ensemble de sommets $T \subset S$ est $G^\prime = \brac{T, A \cap (T \times T)}$.
\end{definition}
\noindent
Comme $G$ est colorié, les voisins de $s$ n'ont pas la même couleur que $s$. Autrement dit, le sous-graphe induit $G^\prime$ ne possède aucun sommets de la même couleur que $s$. Donc $G^\prime$ est $k-$coloriable.

\subsection{}

\textbf{Algorithme de Wigderson :}\\
Voici le principe de l’algorithme de Wigderson. Soit $G$ un graphe à $n$ sommets, et tel que $G$
est 3-coloriable.
\begin{enumerate}
    \item On se donne comme couleur initiale $c = 0$.
    \item Pour chaque sommet s de G pas encore colorié et ayant au moins $\sqrt{n}$ voisins pas encore
coloriés :
    \begin{enumerate}
        \item On $2-$colorie, avec les couleurs $c$ et $c + 1$, le sous-graphe induit par l’ensemble des voisins
de $s$ pas encore coloriés.
        \item On incrémente $c$ du nombre de couleurs utilisées en (a).
    \end{enumerate}
    \item Enfin, on utilise l’algorithme glouton (avec un ordre de numérotation quelconque) pour
colorier, avec des couleurs supérieures ou égales à $c$, le sous-graphe induit par l’ensemble des
sommets pas encore coloriés.
\end{enumerate}

Montrons que l’algorithme de Wigderson appliqué à un graphe 3-coloriable construit toujours un coloriage, et que ce coloriage utilise un nombre de couleur en $\comp{\sqrt{n}}$,
où $n$ est le nombre du sommets du graphe.\\

(i)
Tout d'abord, commençons par montrer que l'algorithe renvoie bien un coloriage. L'algorithme de Wigderson utilise un algorithme de $2-$coloriage et un algorithme glouton. Comme démontré préalablement, ces deux algorithmes sonts corrects. 
De plus, les sous-graphes induits considérés sont coloriés avec des ensembles disjoints de couleurs.
Donc l'algorithme renvoie bien un coloriage.\\

(ii)
Soit $k$ le nombre de sommets traité en 2. Alors l'algorithme colorie au moins $k\times \sqrt{n}$ sommets.
Nous avons donc 
\begin{align*}
    k\sqrt{n} &\leq n\\
    \iff k &\leq \sqrt{n} 
\end{align*}
Le nombre de couleurs utilisées au point 2 est donc au plus $2\sqrt{n}$. Le degré du sous-graphe induit à la fin du point 2 est $\sqrt{n}-1$ (question 4.13). L'algorithme glouton utilise donc au plus $\sqrt{n}$ couleurs pour colorier les sommets restants. Dans le pire cas, on a donc $3\sqrt{n}$ couleurs utilisées, donc $\comp{\sqrt{n}}$ couleurs.

\subsection{}
\lstinputlisting{parts/annexe/code/algorithmedewigderson_15.cpp}

\subsection{}
\lstinputlisting{parts/annexe/code/algorithmedewigderson_16.cpp}

\subsection{}
\lstinputlisting{parts/annexe/code/algorithmedewigderson_17.cpp}

\subsection{}
\lstinputlisting{parts/annexe/code/algorithmedewigderson_18.cpp}

\subsection{}
