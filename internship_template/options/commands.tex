%%%%%%%%%%%%%%%%%LISTE DE COMMANDES%%%%%%%%

%%Commande pour modifier l'environnement


%Commande pour la complexité (moyenne puis pire des cas)
\newcommand{\comptab}[2]{\textcolor{complementaryg}{\textbf{Complexité moyenne : }$\mathcal{O}(#1)$} \hspace{0.5cm} \textcolor{universityred}{\textbf{Complexité maximale : }$\mathcal{O}(#2)$}}
%Commande pour grand O
\newcommand{\comp}[1]{$\mathcal{O}(#1)$}
%Ensembles :
\newcommand{\R}[0]{\mathbb{R}}
\newcommand{\N}[0]{\mathbb{N}}
\newcommand{\Z}[0]{\mathbb{Z}}
\newcommand{\C}[0]{\mathbb{C}}
\newcommand{\G}[0]{\mathcal{G}}
\newcommand{\M}[0]{\mathcal{M}}
\newcommand{\A}[0]{\mathcal{A}}
\newcommand{\T}[0]{\mathcal{T}}
\newcommand{\U}[0]{\mathcal{U}}
\newcommand{\I}[0]{\mathcal{I}}
\newcommand{\card}[1]{\lvert #1 \rvert}
\newcommand{\nor}[0]{\mathrm{norm}}
\newcommand{\norm}[1]{\mathrm{norm}(#1)}
\newcommand{\argmax}[0]{\mathrm{argmax}}
\newcommand{\argmin}[0]{\mathrm{argmin}}



%Commande pour insérer du code (ici en C++)
\newcommand{\code}[1]{\scriptsize \lstinputlisting{#1}}


%Commandes pour les algorithmes en pseudocode

%Probabilities
\newcommand{\prob}[1]{\mathrm{Pr}(#1)}

%Commandes pour 

%Theoreme
\newtheorem{theorem}{\bf Theorem}[section]
\newtheorem{lemma}{\bf Lemma}[section]
\newtheorem{corollary}{\bf Corollary}[section]
\newtheorem{proposition}{\bf Proposition}[section]
\newtheorem{definition}{\bf Definition}[section]
\newtheorem{example}{\bf Example}[section]


%Commandes pour les environnements personnels
\newcommand{\review}[1]{\textcolor{universityred}{#1 -- REVIEW}}
\newcommand{\todo}[1]{\textcolor{red}{\textbf{\Huge TODO : #1}}}

% cref mais avec le titre de la section de destination
% Permet de voir de quoi ça parle pour se dire si oui ou non on veut aller lire la ref 
\newcommand{\niceref}[1]{\cref{#1} -- \nameref{#1}}
